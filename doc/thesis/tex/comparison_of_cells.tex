%!TEX root = ../thesis

\chapter{Comparison of Cells} % (fold)
\label{cha:comparison_of_cells}

\section{Definition of SCC} % (fold)
\label{sec:definition_of_scc}

HiCRep is a mathematical tool introduced in (Yang et al.\cite{yang_hicrep_2017}) for the explicit purpose of comparing Hi-C data sets. It takes as input the contact matrices of two Hi-C data sets (or, more generally, contact matrices of any kind) and outputs a correlation coefficient, that is a number between \(-1\) and \(1\), with larger numbers signifying a stronger similarity between the data sets and vice versa. This is done in a two-step process: first both contact matrices are smoothed to counteract binning-associated problems and then a stratum-adjusted correlation coefficient (SCC) is calculated between these smoothed contact matrices. The exact procedure will be explained in the following section.

\subsection{Smoothing} % (fold)
\label{subsec:smoothing}

Both contact maps are first smoothed using a square uniform filter of width \(2h+1\) for a chosen smoothing parameter \(h \geq 0\). This helps compensating a lack of coverage, that is the fact that not all physical contacts are contained in the contact matrices, a common problem for Hi-C data. Mathematically this filter can be defined as

\[
  X_{ij} = \frac{ \sum_{k=i-h}^{i+h} \sum_{l=j-h}^{j+h} C_{kl} }{ 2h+1 }
\]

where \(C\) is the \(n \times n\) raw contact matrix and \(X\) is the \(n \times n\) smoothed contact matrix. \(C_{ij}\) is defined to be 0 for either \(i\) or \(j\) not in \(\set{1, \dots, n}\).

The uniform filter might seem like an unusual choice compared to other more sophisticated filters, but it has the great advantage of having the representation \( X = L \cdot C \cdot R \), where \(X\) is the smoothed matrix, \(C\) is the raw contact matrix, and \(L\) and \(R\) are upper and lower trigangular matrices respectively. This is useful for computational purposes, especially when using sparse representations of \(C\) and \(X\), which is recommended since contact matrices generally can be quite large, 25,714\(\times\)25,714 in this case, and are mostly empty.

\(h\) is a parameter for the SCC algorithm and thus has to be chosen appropriately. The HiCRep package includes a function called \verb|htrain| (\verb|h_train| in the HiCRep.py python package) that tries to estimate an appropriate h-value heuristically. For a resolution of \(\SI{100}{kbp}\) an example value of \( h = 3 \) is given in the original HiCRep paper, which should be kept in mind as a refence when trying to choose an h-value later.

% subsection smoothing (end)

\subsection{SCC} % (fold)
\label{subsec:scc}

The stratum-adjusted correlation coefficient aims to be a measure of correlation between two random variables \(X\) and \(Y\) that are stratified by a third variable into \(K\) strata \(X_1, \dots, X_K\) and \(Y_1, \dots, Y_K\) respectively. In each stratum there are the stratified random variables \((X_k, Y_k)\) with \(N_k\) observations \( (x_{k,1}, y_{k,1}), \dots, (x_{k,N_k}, y_{k,N_k}) \) each. The pearson correlation coefficient between \(X\) and \(Y\) for the k-th stratum is given by

\[
  \rho_k = \frac{ \mathrm{Cov}(X_k,Y_k) }{ \sqrt{ \mathrm{Var}(X_k) \mathrm{Var}(Y_k)} } = \frac{ \sum_{i=1}^{N_k} (x_{k,i} - \overbar{x}_k) (y_{k,i} - \overbar{y}_k) }{ \sqrt{ \sum_{i=1}^{N_k} x_{k,i} - \overbar{x}_k } \sqrt{ \sum_{i=1}^{N_k} y_{k,i} - \overbar{y}_k } }
\]

The SCC is the weighted average of the pearson correlation coefficients of the different strata:

\[
  \rho_s = \sum_{k=1}^{K} w_k \rho_k
\]

where the weights \(w_k\) are given by

\[
  w_k = \sqrt{ \mathrm{Var}\left( \frac{ \mathrm{Rank}(X_k) }{ N_k } \right) \mathrm{Var}\left( \frac{ \mathrm{Rank}(Y_k) }{ N_k } \right) }
\]

with \( \mathrm{Rank}(X_k) \) and \( \mathrm{Rank}(Y_k) \) being the ranked variables\footnote{\url{https://en.wikipedia.org/wiki/Ranking\#Ranking_in_statistics}} of \(X_k\) and \(Y_k\) respectively. For a thorough derivation of the SCC see the original \mbox{HiCRep} paper\cite{yang_hicrep_2017}, Section \enquote{Derivation of stratum-adjusted correlation coefficient (SCC)}.

% section definition_of_scc (end)

\section{SCC between cells} % (fold)
\label{sec:scc_between_cells}

In Figure~\ref{fig:hic_vs_sim_scc} the SCC between the original Hi-C contact matrices and the contact matrices of the simulation can be seen for each cell. Two beads in the simulation were defined to be in contact if their distance is less than \(3.0\) and contacts were counted across all frames excluding the first five. For all cells the SCC is between 0.68 and 0.80. This is sensible since on one hand, the SCC is expected to be high as the simulated data is based on the Hi-C data, on the other hand it is not surprising that the scores are not perfect since the Hi-C data doesn’t cover all contacts in the real genome whereas the simulation data includes all contacts in the simulated genome. Particularly interesting is that the SCCs for cell 1 and cell 5 are very much in line with those of the other cells, regardless of the problems that arised during the simulation and were discussed in \ref{sub:cell_1} and \ref{sub:cell_5} respectively. This could indicate either that regardless of their problems cell 1 and cell 5 replicated the contacts or their Hi-C data as well or that all simulations replicated the Hi-C data badly so that there is no recognisable difference. This question can unfortunately not be answered here.

The SCC pairwise between the original Hi-C data of all cells can be seen in Table~\ref{tab:scc_hic}. It is immediately very clear that all SCC values (except for those of a cell with itself) are very low, especially compared to the values of \(0.7\) to \(1.0\) obtained in the original HiCRep paper for hESC (human embryonic stem cells) and IMR90 (human lung fibroblast cells) cell lines (Figure~3A in \cite{yang_hicrep_2017}). This might very likely be related to the fact that each of the Hi-C data sets that have been used for this analysis captured only an estimated \(5\%\) of all physical contacts in the actual cell, as detailed in Table~\ref{tab:contact_capture}, but regardless of the reasons, it sets the expectations for similarity between the simulated cells quite low. The SCC for the contact matrices of the simulated genomes can be seen in Table~\ref{tab:scc_sim} and, as was expected, are similarly low to those of the initial Hi-C contact matrices. This strongly suggests that the structure of the simulated genomes have very little relation with each other. This assumption is also backed by the rendered images in \ref{cha:renderings_of_simuated_cells} that show the simulated genomes differ quite strongly in shape, e.g. with some being spherical and others being bean-shaped or obloid, or some of them having some rather big holes inside them making them basically hollow. In summary, the low SCCs are a first strong indicator that the global structure of the different simultated genomes are significantly different.

\begin{table}[ht]
\centering
  \sisetup{ table-alignment-mode=none }
  \caption{SCC calculated pairwise between the contact matrices of the original Hi-C data. Contacts are binned to a size of \(\SI{100000}{bp}\) per bin. Multiple contacts between the same bins are only counted once. The smoothing parameter \(h\) is set to \(7\). The SCC between a cell and itself is always \(1\) and thus omitted.}
  \label{tab:scc_hic}
  \begin{tabular}{S | S S S S S S S S}
   & {Cell 1} & {Cell 2} & {Cell 3} & {Cell 4} & {Cell 5} & {Cell 6} & {Cell 7} & {Cell 8} \\
  \midrule
    {Cell 1} &  {-}  & 0.132 & 0.093 & 0.116 & 0.128 & 0.105 & 0.136 & 0.092 \\
    {Cell 2} & 0.132 &  {-}  & 0.098 & 0.104 & 0.147 & 0.116 & 0.150 & 0.138 \\
    {Cell 3} & 0.093 & 0.098 &  {-}  & 0.062 & 0.104 & 0.093 & 0.101 & 0.092 \\
    {Cell 4} & 0.116 & 0.104 & 0.062 &  {-}  & 0.108 & 0.101 & 0.123 & 0.077 \\
    {Cell 5} & 0.128 & 0.147 & 0.104 & 0.108 &  {-}  & 0.121 & 0.148 & 0.123 \\
    {Cell 6} & 0.105 & 0.116 & 0.093 & 0.101 & 0.121 &  {-}  & 0.145 & 0.097 \\
    {Cell 7} & 0.136 & 0.150 & 0.101 & 0.123 & 0.148 & 0.145 &  {-}  & 0.116 \\
    {Cell 8} & 0.092 & 0.138 & 0.092 & 0.077 & 0.123 & 0.097 & 0.116 &  {-}  \\
  \end{tabular}
\end{table}

\begin{table}[ht]
\centering
  \sisetup{ table-alignment-mode=none }
  \caption{SCC calculated pairwise between the contact matrices of the simulations. Two beads are defined to be in contact if their distance from the centre is less than 3. The smoothing parameter \(h\) is set to \(7\). The SCC between a cell and itself is always \(1\) and thus omitted.}
  \label{tab:scc_sim}
  \begin{tabular}{S | S S S S S S S S}
   & {Cell 1} & {Cell 2} & {Cell 3} & {Cell 4} & {Cell 5} & {Cell 6} & {Cell 7} & {Cell 8} \\
  \midrule
    {Cell 1} &  {-}  & 0.171 & 0.149 & 0.114 & 0.173 & 0.141 & 0.178 & 0.104 \\
    {Cell 2} & 0.171 &  {-}  & 0.164 & 0.141 & 0.183 & 0.184 & 0.215 & 0.184 \\
    {Cell 3} & 0.149 & 0.164 &  {-}  & 0.109 & 0.161 & 0.124 & 0.145 & 0.122 \\
    {Cell 4} & 0.114 & 0.141 & 0.109 &  {-}  & 0.135 & 0.105 & 0.140 & 0.098 \\
    {Cell 5} & 0.173 & 0.183 & 0.161 & 0.135 &  {-}  & 0.148 & 0.194 & 0.143 \\
    {Cell 6} & 0.141 & 0.184 & 0.124 & 0.105 & 0.148 &  {-}  & 0.191 & 0.116 \\
    {Cell 7} & 0.178 & 0.215 & 0.145 & 0.140 & 0.194 & 0.191 &  {-}  & 0.157 \\
    {Cell 8} & 0.104 & 0.184 & 0.122 & 0.098 & 0.143 & 0.116 & 0.157 &  {-}  \\
  \end{tabular}
\end{table}

\FloatBarrier

% section scc_between_cells (end)

\section{RMSD between cells} % (fold)
\label{sec:rmsd_between_cells}

Another way to gauge the similarity of the configurations between the simulated genomes is to calculate RMSDs between the configurations of the cells. There are generally two straightforward approaches that can be taken when calculating RMSDs between the different cells: first,  a reference frame can be chosen for each cell and the RMSDs between these reference frames can be calculated. While this approach is very direct, it does strongly depend on the choice of a good reference frame and is generally much more susceptible to statistical variation between the different frames of a simulation run. The other approach would be to calculate an average trajectory from all (or, as will be discussed, only certain selected) frames in a cell and then calculate the RMSDs between those average trajectories. While this counteracts the statistical problems of using a reference frame, it has a challenge of its own: it only makes sense to take the average of frames that represent the same configuration, i.e. that are already similar to each other. In particular, this not the case for example for the simulation of cell 5, which contains, as discussed in \ref{sub:cell_5}, two configurations, one ground state configuration and one of higher energy, or similarly the simulation of cell 1 which contains besides the ground state frames a number of higher energy frames. To calculate average trajectories, it needs to be ensured that the frames that are averaged over are sufficiently similar and all represent the ground configuration while carefully filtering out the rest.

 For cell 1 the same energy-filtering approach was chosen as in \ref{sub:cell_1} to select only those frames with ground state energy. For cell 5 only frames 39 and onward were included as those represent the ground state frames of the simulation. For the other cells simply the first 5 frames were excluded to avoid the tune-in period as discussed in \ref{sec:simulation_results}. The resulting RMSDs of each frame to both a reference frame, here the last frame of each simulation, and the average trajectory is shown in Figure~\ref{fig:rmsd_last_vs_avg}. There are two major things than can be learned from Figure~\ref{fig:rmsd_last_vs_avg}: first, the RMSD from average is quite consistently smaller than the RMSD from last, implying the averaged trajectories fulfil their purpose as representations of the ground state configuration well. Second, for all cells except cell 1 and cell 5 the mean of the RMSDs from the average trajectory seen in Table~\ref{tab:mean_std_rmsd_from_avg} is between \(0.9\) and \(1.9\), with standard deviations mostly between \(0.08\) and \(0.23\), with the only exception being cell 4 with a slightly higher standard deviation of  \(0.66\). For cell 1 and cell 5 the raw means and standard devations are significantly larger due to the presence of the non ground configuration frames, but after applying the filtering from above the means drop to \(\num{0.68(10)}\) and \(\num{0.88(7)}\) respectively, although this is expected as precisely those frames get filtered out that showed strong deviation from the ground state configuration the averaged trajectory is representing. The unexpected outlier here is cell 4 with a mean RMSDs to average of \(\num{1.55(66)}\) and several pronounced spikes seen in Figure~\ref{fig:rmsd_last_vs_avg}. Looking at the potential energies of the frames with spikes shows that those frames have somewhat elevated potential energy compared to the other frames. Filtering out the nine most promimently spiked frames improves the mean RMSD to average of cell 4 to \(\num{1.36(19)}\), making it fall more in line with the other cells. This improvement is, especially for the standard deviation, significant enough to use this filtering from here on.

\begin{table}[ht]
\centering
  \caption{Mean and standard deviation of the RMSDs of each frame to the average trajectory for each cell simulation.}
  \label{tab:mean_std_rmsd_from_avg}
  \begin{tabular}{l S S S S S S S S}
  \toprule
    Cell & {1} & {2} & {3} & {4} & {5} & {6} & {7} & {8} \\
  \midrule
    RMSD Mean & 3.54 & 0.90 & 0.95 & 1.55 & 5.07 & 1.37 & 1.88 & 1.78 \\
    RMSD Std & 2.46 & 0.16 & 0.23 & 0.66 & 2.58 & 0.21 & 0.12 & 0.08 \\
  \bottomrule
  \end{tabular}
\vspace{0.4cm}
\end{table}

 Using the filterings described above to generate the average trajectory for each cell, the RMSDs between each of them can be seen in Table~\ref{tab:rmsd_avg_between_cells}. Except for the self-RMSDs they are all between \(18.4\) and \(25.0\), implying the average whole genome trajectories for each cell simulation are significantly different, affirming the same conclusion made in \ref{sec:scc_between_cells} using the genome level SCC.

\begin{table}[ht]
\centering
  \caption{RMSDs between average trajectories of each cell. Each average trajectory is generated from the ground state frames of the simulations. The RMSD between a cell and itself is always \(0\) and thus omitted.}
  \label{tab:rmsd_avg_between_cells}
  \begin{tabular}{S | S S S S S S S S}
  % \toprule
     & {Cell 1} & {Cell 2} & {Cell 3} & {Cell 4} & {Cell 5} & {Cell 6} & {Cell 7} & {Cell 8} \\
  \midrule
    {Cell 1} &  {-} & 19.4 & 19.8 & 18.4 & 19.2 & 19.6 & 23.7 & 24.3 \\
    {Cell 2} & 19.4 &  {-} & 18.9 & 18.4 & 19.6 & 20.1 & 24.5 & 24.1 \\
    {Cell 3} & 19.7 & 18.9 &  {-} & 18.4 & 19.5 & 19.9 & 22.1 & 23.9 \\
    {Cell 4} & 18.4 & 18.4 & 18.4 &  {-} & 19.1 & 19.3 & 24.5 & 24.5 \\
    {Cell 5} & 19.2 & 19.6 & 19.5 & 19.1 &  {-} & 19.6 & 25.0 & 23.4 \\
    {Cell 6} & 19.7 & 20.1 & 19.9 & 19.3 & 19.6 &  {-} & 23.9 & 22.6 \\
    {Cell 7} & 23.7 & 24.5 & 22.1 & 24.5 & 25.0 & 23.9 &  {-} & 24.4 \\
    {Cell 8} & 24.3 & 24.1 & 23.9 & 24.5 & 23.4 & 22.6 & 24.4 &  {-} \\
  % \bottomrule
  \end{tabular}
\end{table}

% section rmsd_between_cells (end)

% chapter comparison_of_cells (end)