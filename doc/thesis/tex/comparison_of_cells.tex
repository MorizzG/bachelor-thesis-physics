%!TEX root = ../thesis

\chapter{Comparison of Cells} % (fold)
\label{cha:comparison_of_cells}

\section{Definition of SCC} % (fold)
\label{sec:definition_of_scc}

HiCRep is a mathematical tool introduced in \cite{yang_hicrep_2017} for the explicit goal of comparing Hi-C data sets. It takes as input the contact matrices of two Hi-C data sets and outputs a correlation coefficient, that is a number between \(-1\) and \(1\), with higher numbers signifying a stronger similarity between the data sets and vice versa. This is done in a two-step process: first both contact matrices are smoothed to counter binning-associated problems and then a stratuma djusted correlation coefficient (SCC) is calculated between these smoothed contact matrices. The exact procedure will be explained in the following section

\subsection{Smoothing} % (fold)
\label{subsec:smoothing}

Both contact maps are first smoothed using a uniform filter of width \(2h+1\) for a chosen smoothing parameter \(h \geq 0\). This helps compensating a lack of coverage, that is the fact that not all actual contacts are contained in the contact matrices, something that is common for Hi-C data. Mathematically this filter is defined as

\[
  X_{ij} = \frac{ \sum_{k=i-h}^{i+h} \sum_{l=j-h}^{j+h} C_{kl} }{ 2h+1 }
\]

where \(C\) is the \(n \times n\) contact matrix and \(X\) is the \(n \times n\) smoothed matrix. \(C_{ij}\) is defined to be 0 for either \(i\) or \(j\) not in \(1 \dots n\).

The uniform filter might seem like an unusual choice compared to other more sophisticated filters, but has the great advantage of having the representation \( X = L \cdot C \cdot R \), where \(X\) is the smoothed matrix, \(C\) is the contact matrix, and \(L\) and \(R\) are upper and lower trigangular matrices respectively. This comes in handy especially when using sparse representations of \(C\) and \(X\), which is very much recommended since contact matrices generally can be quite large and are largely empty.

\(h\) is a parameter for the SCC algorithm and thus has to be chosen appropriately. The HiCRep package includes a function called \verb|htrain| (\verb|h_train| in the HiCRep.py python package) that tries to estimate an appropriate h-value heuristically. For our resolution of \(\SI{100}{kbp}\) an example value of \( h = 3 \) is given in the original HiCRep paper, which should be kept in mind as a refence later when trying to choose an h-value for our own data.

% subsection smoothing (end)

\subsection{SCC} % (fold)
\label{subsec:scc}

The stratum adjusted correlation coefficient aims to be a measure of correlation between two random variables \(X\) and \(Y\), stratified by a third variable into \(K\) strata \(X_1, \dots, X_K\) and \(Y_1, \dots, Y_K\) respectively. In each stratum we have the stratified random variables \((X_k, Y_k)\) with \(N_k\) observations \( (x_{k,1}, y_{k,1}), \dots, (x_{k,N_k}, y_{k,N_k}) \) each. The pearson correlation coefficient between \(X\) and \(Y\) for the k-th stratum is given by

\[
  \rho_k = \frac{ \mathrm{Cov}(X_k,Y_k) }{ \sqrt{ \mathrm{Var}(X_k) \mathrm{Var}(Y_k)} } = \frac{ \sum_{i=1}^{N_k} (x_{k,i} - \overbar{x}_k) (y_{k,i} - \overbar{y}_k) }{ \sqrt{ \sum_{i=1}^{N_k} x_{k,i} - \overbar{x}_k } \sqrt{ \sum_{i=1}^{N_k} y_{k,i} - \overbar{y}_k } }
\]

The SCC is the weighted average of the pearson correlation coefficients

\[
  \rho_s = \sum_{k=1}^{K} w_k \rho_k
\]

where the weights \(w_k\) are

\[
  w_k = \sqrt{ \mathrm{Var}\left( \frac{ \mathrm{Rank}(X_k) }{ N_k } \right) \mathrm{Var}\left( \frac{ \mathrm{Rank}(Y_k) }{ N_k } \right) }
\]

with \( \mathrm{Rank}(X_k) \) and \( \mathrm{Rank}(Y_k) \) being the ranked variables\footnote{\url{https://en.wikipedia.org/wiki/Ranking\#Ranking_in_statistics}}. For a thorough derivation of the SCC see the original HiCRep paper\cite{yang_hicrep_2017}, Section \enquote{Derivation of stratum-adjusted correlation coefficient (SCC)}.

% section definition_of_scc (end)

\section{SCC between cells} % (fold)
\label{sec:scc_between_cells}

In Figure~\ref{img:hic_vs_sim_scc} the SCC between the original Hi-C contact matrices and the contact matrices of the simulation can be seen for each cell. For all cells the SCC is between 0.68 and 0.80. This is sensible since on one hand, the SCC is expected to be high as the simulated data is based on the Hi-C data, on the other hand it is not surprising that the scores are not perfect since the Hi-C data doesn’t cover all contacts in the real genome whereas the simulation data includes all contacts in the simulated genome. Particularly interesting is that the SCCs for cell 1 and cell 5 are very much in line with those of the other cells, regardless of the problems that arised during the simulation and were discussed in \ref{ssub:cell_1} and \ref{ssub:cell_5} respectively. This could indicate either that regardless of the problems cell 1 and cell 5 replicated the contacts or their Hi-C data as well as the others or that replicated the Hi-C data badly so that there is no recognisable difference.

The SCC for the original Hi-C data of all cells can be seen in Table~\ref{tab:scc_hic}. It is immediately very clear that all SCC values (except for those of a cell with itself) are very low, especially compared to the values of \(0.7\) to \(1.0\) obtained in the original HicRep paper for hESC (human embryonic stem cells) and IMR90 (human lung fibroblast cells) cell lines (Figure~3A in \cite{yang_hicrep_2017}). This might very likely be related to the fact that each Hi-C data set captured only about \(5\%\) of all contacts, as detailed in Table~\ref{tab:contact_capture}, but regardless of the reasons, it sets the expectations for comparability between the cells quite low. The SCC for the simulated contact matrices can be seen in Table~\ref{tab:scc_sim} and, as expected, are similarly low. This stronly suggests that the simulated cells have very little relation with each other. This assumption is also backed by the rendered images in \ref{cha:renderings_of_simuated_cells} that show the simulated genomes can differ quite strongly in shape, e.g. with some being spherical and others being bean-shaped or obloid, or some of them having some rather big holes inside them making them basically hollow.

\begin{table}[ht]
\centering
  \sisetup{ table-alignment-mode=none }
  \caption{\textcolor{red}{SCC between HiC contact matrices. h = 7}}
  \label{tab:scc_hic}
  \begin{tabular}{S | S S S S S S S S}
   & {Cell 1} & {Cell 2} & {Cell 3} & {Cell 4} & {Cell 5} & {Cell 6} & {Cell 7} & {Cell 8} \\
  \midrule
    {Cell 1} &  {-}  & 0.132 & 0.093 & 0.116 & 0.128 & 0.105 & 0.136 & 0.092 \\
    {Cell 2} & 0.132 &  {-}  & 0.098 & 0.104 & 0.147 & 0.116 & 0.150 & 0.138 \\
    {Cell 3} & 0.093 & 0.098 &  {-}  & 0.062 & 0.104 & 0.093 & 0.101 & 0.092 \\
    {Cell 4} & 0.116 & 0.104 & 0.062 &  {-}  & 0.108 & 0.101 & 0.123 & 0.077 \\
    {Cell 5} & 0.128 & 0.147 & 0.104 & 0.108 &  {-}  & 0.121 & 0.148 & 0.123 \\
    {Cell 6} & 0.105 & 0.116 & 0.093 & 0.101 & 0.121 &  {-}  & 0.145 & 0.097 \\
    {Cell 7} & 0.136 & 0.150 & 0.101 & 0.123 & 0.148 & 0.145 &  {-}  & 0.116 \\
    {Cell 8} & 0.092 & 0.138 & 0.092 & 0.077 & 0.123 & 0.097 & 0.116 &  {-}  \\
  \end{tabular}
\end{table}

\begin{table}[ht]
\centering
  \sisetup{ table-alignment-mode=none }
  \caption{\textcolor{red}{SCC between simulated contact matrices. h = 7}}
  \label{tab:scc_sim}
  \begin{tabular}{S | S S S S S S S S}
   & {Cell 1} & {Cell 2} & {Cell 3} & {Cell 4} & {Cell 5} & {Cell 6} & {Cell 7} & {Cell 8} \\
  \midrule
    {Cell 1} &  {-}  & 0.171 & 0.149 & 0.114 & 0.173 & 0.141 & 0.178 & 0.104 \\
    {Cell 2} & 0.171 &  {-}  & 0.164 & 0.141 & 0.183 & 0.184 & 0.215 & 0.184 \\
    {Cell 3} & 0.149 & 0.164 &  {-}  & 0.109 & 0.161 & 0.124 & 0.145 & 0.122 \\
    {Cell 4} & 0.114 & 0.141 & 0.109 &  {-}  & 0.135 & 0.105 & 0.140 & 0.098 \\
    {Cell 5} & 0.173 & 0.183 & 0.161 & 0.135 &  {-}  & 0.148 & 0.194 & 0.143 \\
    {Cell 6} & 0.141 & 0.184 & 0.124 & 0.105 & 0.148 &  {-}  & 0.191 & 0.116 \\
    {Cell 7} & 0.178 & 0.215 & 0.145 & 0.140 & 0.194 & 0.191 &  {-}  & 0.157 \\
    {Cell 8} & 0.104 & 0.184 & 0.122 & 0.098 & 0.143 & 0.116 & 0.157 &  {-}  \\
  \end{tabular}
\end{table}

% section scc_between_cells (end)

% chapter comparison_of_cells (end)