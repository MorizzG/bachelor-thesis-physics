%!TEX root = ../thesis

\chapter{Conclusion} % (fold)
\label{cha:conclusion}

Using the protocol defined in Wettermann et al.\cite{wettermann_minimal_2020}, a total of eight cells were simulated using a molecular dynamics simulation with a number of predefined bonds and contacts derived from Hi-C data sets. Both the potential energies and the distance distribution of the predefined bonds and contacts show high agreement with (Wettermann et al.\cite{wettermann_minimal_2020}) and suggest that in most cases the simulation reaches a stable ground configuration.

The two major exceptions, cell 1 and cell 5, were analysed more in-depth and at least in the case of cell 5 the problematic features of the simulation could be resolved to be merely an artefact of that particular simulation run, with reruns of the simulation of that particular cell falling very much in line with the results of the other cells.

As for cell 1, while it was shown that a ground configuration still existed, even though it was not stable, the instability of the configuration was consistently observable across multiple simulation runs. A deeper reason for why this effect arises only in cell 1 and not the other cells could not be given, and could be a question for further research.

The analysis of cell 5 on the other hand showed the possibility of semi-stable configurations to exist with higher energies compared to the ground state energy, that appear in some simulation runs, but not in others. Considering the fact that this semi-stable state disappeared by transitioning into the ground state configuration after more simulation cycles suggests that in this case the effect might not be biologically relevant as it disappeared on its own, but implies that the existence of relevant configurations besides the ground state is certainly possible. This raises the possibility of misfolded genome configurations, like misfolded proteins, that could change the transcription behaviour of a cell.

The simulation results from the different cells were compared pairwise for whole-genome similarity using both SCC and RMSD as similarity measures. The SCC between the initial Hi-C contact matrices and the contact matrices of the simulations showed decent correspondence, but considering the fact the Hi-C data set contains only an estimated \(5\%\) of all the contacts in the physical cells, these scores are still reasonable. This implies that the simulation model managed to reproduce the Hi-C data set reasonably well.

The global genome structures of the simulations for each cell were compared using both SCC and RMSD as a measure of comparison. Both measures unquestionably implied that the correspondence between global structures of each cells were significantly different, with the SCCs being in a range of \(0.10\) to \(0.22\) and the RMSDs being in a range of \(18.4\) and \(25.0\).

Similarly, for comparisons of individual chromosomes from the entire genome simulations between cells no significant similarity could be found, with the RMSD for no chromosome and no cell combination falling below \(4.7\). This implies that, just like on the whole genome level, the structure of the individual chromosomes varies significantly between cells.

Furthermore, chromosome 1 and chromosome 19 were also simulated in isolation to determine how much the structure of each chromosome depends on the influence of the other chromosomes opposed to the intrinsic interactions within the chromosome itself. For both chromosomes the results quite clearly indicated that the influence of the other chromosomes is not negligible, with significant differences arising between the chromosomes simulated in isolation and in the context of the entire cell.

In summary, both in the entire genome structure as well as the structure of the individual chromosomes no significant similarity could be found between any of the cells in any way. The straightforward explanation for this would be that exact structural resemblences on high levels simply do not exist between cells, even of in cells of the same cell line, which is in accordance with the findings in (Stevens et al.\cite{stevens_3d_2017}). In the same way of thinking, some current research seems to be focusing more on other kinds of structures such as A/B compartments, topological-associated domains, and loops\cite{stevens_3d_2017}. The chance of finding simple global patterns in genome structures seems to be rather slim at this point.

% chapter conclusion (end)

\vfill

The source code used for the analysis can be found at \url{https://github.com/Lego1120/chromatin-structure}.