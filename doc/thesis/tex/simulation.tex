%!TEX root = ../thesis

\chapter{Simulation} % (fold)
\label{cha:simulation}

\section{Model and Simulation Protocol} % (fold)
\label{sec:model_and_simulation_protocol}

The simulation protocol for the entire genome simulations was mostly carried over from (Wettermann et al.\cite{wettermann_minimal_2020}). Additionally simulations of single chromosomes were carried out; the protocol for these simulations is identical to that of the entire genome, except only the beads of the chromosome in question were modelled. The simulation is a molecular dynamics simulation using the HOOMD-blue\cite{anderson_hoomd-blue_2020} toolkit. It utilises a Langevin integrator (\verb|hoomd.md.integrate.langevin|) with a timestep of \(dt=0.001\), a temperature of \(kT = 1.0\), and a drag coefficient of \(\gamma = 1.0\). The neighbour list is a BVH tree neighbour list \cite{howard_efficient_2016} \cite{howard_quantized_2019} that was originally chosen as it scales
with particle number as opposed to the system volume\cite{wettermann_minimal_2020}.

Each chromosome is modelled as beads on a string, where each bead represents a bin of \(\SI{100000}{bp}\). This is the same resolution as was chosen in (Wettermann et al.\cite{wettermann_minimal_2020}) and represents a compromise between the resolution of the simulation result and the quality of the contact data that is available: a higher resolution, i.e. a smaller bin size of for example \(\SI{40000}{bp}\), would increase the resolution of the simulated structure and enable us to see smaller structures, but at the same time would spread the fixed number of contacts across a higher number of beads. This would make in particular the effect of having captured only a fraction of all possible contacts in the cell using Hi-C more prominent, which is estimated to be around \(5\%\) for each cell as seen in Table~\ref{tab:contact_capture}. Vice versa, decreasing the bin size would help mitigate the partial capture of contacts, but limit the spatial resolution of the simulated structure. With this chosen resolution of \(\SI{100000}{bp}\) per bin the genome is represented by 20 chains varying in length between 500 and 2,000 beads each (the exact lengths for each chromosome can be found in Table~\ref{tab:chrom_lengths}), or 25,714 beads in total. This does not represent the entirety of the mouse genome, whose length is approximately \(\SI{2632}{Mbp}\)\footnote{Mouse Genome Assembly GRCm39 from \url{https://www.ncbi.nlm.nih.gov/grc/mouse/data}, visited on 26.02.2022}, or 26,321 beads at a resolution of \(\SI{100000}{bp}\) per bead. The reason for this difference is that beads at the boundary of a chromosome that had no contact in any of the eight cells were dropped from the simulation, since their impact was assumed to be only negligible. Boundary beads that had contacts in some cells but not others were kept in the simulation of all cells in order to keep the simulation data consistent across cells.

The model is based on a generic bead-spring polymer model in which three kinds of bonds are defined. The first two kinds of bonds are harmonic bonds between two beads of the general form

\[
  V(r) = \frac{2} \kappa \left( r - r_0 \right)^2
\]

where \(\kappa\) is the force constant determining the stiffness of the bond, which is fixed at \(\kappa = 2000\) for all harmonic bonds, and \(r_0\) is the preferred bond distance. The first kind of harmonic bonds are the backbone bonds connecting adjacent beads in each chromosome; for these bonds the preferred distance is set to \(r_0 = 1.0 \). The other kind of harmonic bonds in the simulation are the predefined contacts derived from the Hi-C data set from (Stevens et al.\cite{stevens_3d_2017}). Here the preferred bond distance is set a little larger compared to the backbone at \(r_0 = 1.5\) in accordance with (Wettermann et al.\cite{wettermann_minimal_2020}).

The third kind of bond is a Gaussian pair potential of the form

\[
  V(r) = \begin{cases}
    \epsilon \exp \left[ - \frac{2} \left( \frac{r}{\sigma} \right)^2 \right] & r < r_\text{cut} \\
    0 & r \geq r_\text{cut}
  \end{cases}
\]

between all beads in the simulation designed to push all non-bonded beads away from each other. This potential is used in 2 forms in different parts of the simulation: a full form with \(\sigma = 1.0\) and \(r_\text{cut} = 3.5\) and a reduced form with \(\sigma = 0.1\) and \(r_\text{cut} = 0.4\); in both cases \(\epsilon = 100\). On one hand, this mimics the fact that at physiological conditions DNA molecules are negatively charged and thus repels each other. On the other hand it represents an excluded volume potential that pushes all beads away from each other, which is  quite significant for the emergence of chromosomal territories\cite{wettermann_minimal_2020}. An overview of all the potentials in the simulation can be seen in Figure~\ref{fig:potentials}.

\begin{figure}[ht]
\centering
  \includegraphics[width=\figwidth]{potentials.png}
  \caption{Potentials used in the simulation. Bonds and contacts are harmonic potentials of the form \(V(r) = 1000 (r - r_0)^2\), with bonds having an \(r_0\) of 1.0 and contacts having an \(r_0\) of 1.5. Gauss potentials are of the form \(V(r) = 100 \exp\left[- \frac{2} \left( \frac{r}{\sigma} \right)^2 \right]\) for \(r\) smaller than \(r_\text{cut}\) and \(0\) for \(r\) greater than the cutoff value \(r_\text{cut}\).}
  \label{fig:potentials}
\end{figure}

The system is initialised by distributing all the beads randomly throughout the simulation box (uniform distribution, using \verb|numpy.uniform.random|\cite{harris_array_2020}). The bonds are set and the simulation is repeatedly cycled through the following steps:

\begin{itemize}[label=\(\bullet\)]
  \item 80,000 time steps with no excluded volume potential
  \item 50,000 time steps with reduced volume potential
  \item 50,000 time steps with full excluded volume potential
\end{itemize}

Bonds and contacts are active at all of those steps. After each cycle the current state is saved to a gsd trajectory file. These saved states will be referred to in the following as \textbf{frames}. Frames whose trajectories are similar are said to have the same \textbf{configuration}. These steps were repeated in each simulation for a total of 105 cycles. The first few cycles have to be discarded as the system takes some time to find its ground state, although certain problems can arise here that will be discussed later in \ref{sec:problems_with_the_simulation}.

% section model_and_simulation_protocol (end)

\section{Simulation results} % (fold)
\label{sec:simulation_results}

Each simulation yields 105 sequential frames, i.e. the simulation state is not reset after each simulation cycle, but instead the final state of the last cycle is the initial state of the next cycle. This has the advantage of giving the system time to tune in, but also the disadvantage of the possibility that certain end configurations will never be reached in a particular simulation run after it has tuned in to a different locally minimal configuration. The potential energies of each frame for the simulation run of cell 2 can be seen in Figure~\ref{fig:potential_energy_cell2}. The first two frames show a potential energy significantly larger than the later ones, then the system quickly converges to a potential energy of about \(\num{7950000}\) and shows only small deviations of less than \(1 \%\) of the mean. Thus both the length of the settling period and the potential energy of the ground configuration match (Wettermann et al.\cite{wettermann_minimal_2020}) extremely well. To minimise the effect of the settling period the first 5 frames of each simulation run will generally be excluded in all subsequent analyses where the data is combined over all frames such as averages or standard deviations.

Furthermore, Figure~\ref{fig:distance_pdf_cell2} shows the distance distribution of the bonds and predetermined contacts combined across all frames of the simulation of cell 2. The results are again very similar to (Wettermann et al.\cite{wettermann_minimal_2020}), with both peaks and means shifted slightly to the right of the respective preferred bond length of \(1.0\) and \(1.5\). The \(99.73\)th quartile is at \(1.71\) for bonds and \(2.42\) for contacts, showing that a substantial portion of the bonds and predetermined contacts are enforced reasonably well.

\begin{figure}[ht]
\centering
  \includegraphics[width=12cm]{potential_energy_cell2.png}
  \caption{Potential energy of each frame in the simulation of cell 2.}
  \label{fig:potential_energy_cell2}
\end{figure}

\begin{figure}[ht]
\centering
  \includegraphics[width=12cm]{distance_pdf_cell2.png}
  \caption{Distance distributions for bonds and predetermined contacts over all frames in the simulation of cell 2 with the means marked by a vertical line. The mean of the bond distances is at \(\num{1.10}\) and the mean of the contact distances is at \(\num{1.61}\).}
  \label{fig:distance_pdf_cell2}
\end{figure}

For the other cells the situation is generally similar, except for cell 1 and cell 5, which will be discussed more in-depth in \ref{sec:problems_with_the_simulation}. The coefficients of variation, i.e. the standard deviations divided by the means, of potential energies are between \(0.3\%\) and \(1.8\%\), showing the minimum energy configuration to be quite stable in these cells. Also neither bonds nor contacts are overly overstretched in any cells, including cell 1 and cell 5. A complete overview of potential energy coefficients of variation and means and \(99.73\)th percentiles for bonds and predefined contacts can be found in Table~\ref{tab:simulation_pe_dists}.

Renderings of all simulated cells can be seen in Appendix~\ref{cha:renderings_of_simuated_cells}. A few things can be notes from these images visually. First, most simulated genomes have a quite regular, spherical shape, but in particular cell 3, cell 8, and the higher energy configuration of cell 5 (which will be discussed in more detail in \ref{sec:problems_with_the_simulation}) show clear differences from this. Cell 3 and cell 8 have a more elongated, bean-like shape, as can be seen in Figure~\ref{img:cell3_frame104_scene1} for cell 3 and in Figure~\ref{img:cell8_frame104_scene2} for cell 8. The higher energy configuration of cell 5 on the other hand has a more obloid, donut-like shape. Also notable is the fact that cell 6 and cell 7 are hollow as can be seen for example in Figure~\ref{img:cell6_frame104_scene1} for cell 6 and Figure~\ref{img:cell7_frame104_scene1} for cell 7. More renderings of all simulated cells can be seen in Appendix \ref{cha:renderings_of_simuated_cells}.

% section simulation_results (end)

\section{Problems with the simulation} % (fold)
\label{sec:problems_with_the_simulation}

While the simulations of most cells quickly reached a stable ground state configuration, the simulations of cell 1 and of cell 5 showed significant deviation from the expected results, warranting further investigation.

\subsection{Cell 1} % (fold)
\label{sub:cell_1}

In cell 1, a ground state is reached very quickly, but this ground state is very unstable, as indicated by the comparably high coefficient of variation of the potential energy of 8.51\% in Table~\ref{tab:simulation_pe_dists} and seen in Figure~\ref{fig:potential_energy_cell1}. The RMSDs of all frames with respect to the last frame as seen in Figure~\ref{fig:rmsd_cell1} shows that some of these frames have rather similar configurations while others are very different. One hypothesis would be that some frames represent a ground state configuration while others are higher energy states. This can be tested by selecting for all frames with low energy, defined by being below some cutoff energy, and checking how the RMSDs for those low-energy frames behaves. A cutoff energy of \(\num{1.495e7}\), visually displayed in Figure~\ref{fig:potential_energy_cell1} as the orange line, was chosen as it captures most of the frames that can be identified visually as being low energy, while excluding all that deviate strongly from this energy baseline. This selects a total of 52 frames, which have been marked in Figure~\ref{fig:rmsd_cell1} by red dot. As can be seen very clearly, these low energy frames do in fact have a low RMSD of \(\num{1.00(1)}\) with respect to the last frame. This confirms that while it is very unstable, this simulation of cell 1 does in fact have a ground state configuration, and it can be filtered for by selecting for those frames with a low energy. To check if this ground state instability is an intrinsic property of the Hi-C contact data for cell 1 or merely a random artefact of this particular simulation run, cell 1 was simulated two more times. The potential energies for those repeated simulations can be seen in Figure~\ref{fig:potential_energy_cell1_1} and Figure~\ref{fig:potential_energy_cell1_2} in the appendix and clearly show the same pattern of instability, leading to the conclusion that this instability is in fact a consequence of the predetermined contacts for cell 1.

\begin{figure}[ht]
\centering
  \includegraphics[width=10.5cm]{potential_energy_cell1.png}
  \caption{Potential energies of all frames in the simulation of cell 1. Cutoff energy is set at \(\num{1.495e7}\), with all frames lower than this threshold being defined as \enquote{low-energy frames}.}
  \label{fig:potential_energy_cell1}
\end{figure}

\begin{figure}[ht]
\centering
  \includegraphics[width=10.5cm]{rmsd_cell1.png}
  \caption{RMSDs of cell 1 with respect to last frame. Low-energy frames, as defined in Figure~\ref{fig:potential_energy_cell1} to be frames with a potential energy lower than \(\num{1.495e7}\), have their RMSD marked by a red dot.}
  \label{fig:rmsd_cell1}
\end{figure}

% subsection cell_1 (end)

\FloatBarrier

\subsection{Cell 5} % (fold)
\label{sub:cell_5}

Figure~\ref{fig:potential_energy_cell5_all} shows the potential energies of all frames in the simulation of cell 5 in blue. As can be seen, after the zeroth frame, which is of higher energy as is typical for the transient phase, the potential energy drops to and stabilises around a value of \(\num{9.805(43)e6}\) for frames 1 through 38. But then for frame 39 and the following ones until the end of the simulation, the energy drops down again to a significantly lower value of \(\num{8.220(8)e6}\). As seen in Figure~\ref{fig:potential_energy_cell5_all}, this is a reduction in both the potential energy itself as well as the deviation in potential energy, from a coefficient of variation of \(0.5 \%\) down to \(0.1 \%\), signifying that the second configuration is both energetically more favourable as well as very stable. The fact that the first configuration was held for 38 frames though means that the existance of semi-stable configurations apart from the ground state configuration is possible. To examine this effect further, cell 5 was simulated 2 more times. The resulting potential energies, together with the potential energies of the first simulation, can be seen in Figure~\ref{fig:potential_energy_cell5_all}. Clearly the second and third simulation run show the same pattern that most other cells exhibit: the first one or two frames have a higher potential energy, and then it drops into a stable ground state configuration where it remains for the rest of the simulation. The very similar potential energies of the ground state configurations in all three simulations suggests that these ground state configurations are in fact the same across all simulations. An analysis of the RMSDs of each frame in all three simulations from the last frame of the first simulation can be seen in Figure~\ref{fig:rmsd_cell5_all} and clearly confirms that the ground states of all three simulations are in fact the same one. This means that while the simulation of cell 5 does have a single stable ground state configuration, it also has a semi-stable configuration that arises in some, but not all simulation runs.

% \begin{figure}[ht]
% \centering
%   \includegraphics[width=\figwidth]{potential_energy_cell5.png}
%   \caption{Potential energies of all frames in the (first) simulation of cell 5.}
%   \label{fig:potential_energy_cell5}
% \end{figure}

\begin{figure}[ht]
\centering
  \includegraphics[width=\figwidth]{potential_energy_cell5_all.png}
  \caption{Potential energies of all frames in all three simulations of cell 5.}
  \label{fig:potential_energy_cell5_all}
\end{figure}

\begin{figure}[ht]
\centering
  \includegraphics[width=\figwidth]{rmsd_cell5_all.png}
  \caption{RMSD of all three simulations of cell 5 with respect to the last frame of the first simulation.}
  \label{fig:rmsd_cell5_all}
\end{figure}

% subsection cell_5 (end)

% section problems_with_the_simulation (end)


% chapter simulation (end)