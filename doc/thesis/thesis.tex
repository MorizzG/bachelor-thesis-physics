%_______________________________________________________________________________
%class
%_______________________________________________________________________________
%\documentclass[a4paper,11pt,onecolumn,final,german,openbib]{scrbook}
\documentclass[a4paper,11pt,oneside,final,english,toc=bib]{scrbook}

% fonts

\usepackage{fontspec}
\setmainfont{Libre Baskerville} % Libre Baskerville - Palatino Linotype - Constantia
\setsansfont{Helvetica} % Arial
\setmonofont{Consolas} % Hack
%_______________________________________________________________________________
% page borders
%_______________________________________________________________________________
\addtolength{\headheight}{2cm}
%\addtolength{\topmargin}{2cm}
\setlength{\oddsidemargin}{1.0cm}
\setlength{\evensidemargin}{0.5cm}
\setlength{\textwidth}{14.3cm}
\setlength{\parindent}{0mm}

%_______________________________________________________________________________
% packages
%_______________________________________________________________________________
% \usepackage{german}
\usepackage{babel}[uk-english]
% \usepackage{textcomp}
\usepackage{csquotes}

\usepackage{mathtools}
\usepackage{amssymb}
% \usepackage{physics}
% \usepackage[separate-uncertainty=true,per-mode=power]{siunitx} % ,per-mode=fraction 
% \usepackage[version=4]{mhchem}
% \usepackage{dsfont}
% \usepackage{slashed}

\usepackage{graphicx}
\usepackage{subfigure}
% \usepackage{float}
\usepackage{caption}
% \usepackage{subcaption}

% \usepackage{url}
\usepackage{hyperref}
% \usepackage{enumerate}

\usepackage{booktabs}
% \usepackage{multirow}

% \usepackage{appendix}

\usepackage[style=ieee]{biblatex} % ,sorting=none style=h-physrev3 defernumbers=true
\bibliography{Physik_thesis_Bachelorarbeit}


\graphicspath{ {./images/} }

\hypersetup{
  colorlinks=true,
  linkcolor=blue,
  citecolor=green
}

\captionsetup[figure]{labelfont=bf,format=hang,labelsep=period,margin=1cm}
\captionsetup[table]{labelfont=bf,format=hang,labelsep=period,margin=1cm} % ,labelsep=newline

\renewcommand{\arraystretch}{1.2}

\parindent10mm

\usepackage{mgscience}

%_______________________________________________________________________________
% bold fonts for headings
%_______________________________________________________________________________
\font\afont=cmssbx10 scaled \magstep5     % for the title
\font\bfont=cmssbx10 scaled \magstep4     % for chapter headings
\font\cfont=cmssbx10 scaled \magstep3
\font\dfont=cmssbx10 scaled \magstep2     % for section headings and author name
\font\efont=cmssbx10 scaled \magstephalf

%_______________________________________________________________________________
% index depth
%_______________________________________________________________________________
\setcounter{secnumdepth}{3}
\setcounter{tocdepth}{3}

%_______________________________________________________________________________
% new commands
%_______________________________________________________________________________
% \newcommand{\demi}{\frac{1}{2}}

%_______________________________________________________________________________
% renewed commands
%_______________________________________________________________________________
% \renewcommand{\topfraction}{1.}       % this is important for figure placement
% \renewcommand{\bottomfraction}{1.}
\makeatletter
\renewcommand\paragraph{\@startsection{paragraph}{4}{\z@}%
  {-3.25ex\@plus -1ex \@minus -.2ex}%
  {1.5ex \@plus .2ex}%
  {\normalfont\normalsize\bfseries}
}
\makeatother

%_______________________________________________________________________________
% special words, hyphenation
%_______________________________________________________________________________
% \hyphenation{Ba-che-lor-ar-beit}

\pagestyle{empty}
\pagestyle{headings}
%for changing the style on a specific page use \thispagestyle{e.g., empty}

%_______________________________________________________________________________
%_______________________________________________________________________________
\begin{document}
\pagenumbering{roman}

%_______________________________________________________________________________
\begin{titlepage}
  \vspace*{6mm}
  \begin{center}
     {\afont Title of the Bachelor Thesis}
     \\[3.5cm]
     {\large von}
     \\[3.5cm]
     {\dfont Moritz Gmeiner}
     \\[1.5cm]
     {\dfont Supervisor: PD Dr. Peter Virnau}
     \\[2cm]
     % {\large Bachelorarbeit in Physik \/\\
     %    vorgelegt dem Fachbereich Physik, Mathematik und Informatik (FB 08) \/\\
     %    der Johannes Gutenberg-Universität Mainz \/\\
     %    am 1. April 2012}
     {\large Bachelor Thesis in Physics \/\\
        presented to the faculty physics, mathematics, and computer science (FB 08) \/\\
        of the Johannes Gutenberg University Mainz \/\\
        \today}
   \end{center}
   \vfill
   1. Reviewer: PD Dr. Peter Virnau \\	
   2. Reviewer: Prof. Dr. Friederike Schmid \\
   \vfill
\end{titlepage}

\thispagestyle{empty}
Ich versichere, dass ich die Arbeit selbstständig verfasst und keine 
anderen als die angegebenen Quellen und Hilfsmittel benutzt sowie 
Zitate kenntlich gemacht habe.
\\
\\[3.5cm] 
Mainz, den \today
\vfill
\noindent 
Johanna Musterfrau\\
KOMET\\
Institut für Physik\\
Staudingerweg 7\\
Johannes Gutenberg-Universität
D-55099 Mainz\\
{ \texttt{Johanna.Musterfrau@uni-mainz.de} }

%_______________________________________________________________________________
% \renewcommand\contentsname{Inhaltsverzeichnis}
% \renewcommand\figurename{Abbildung}
% \renewcommand\tablename{Tabelle}

\tableofcontents
\clearpage

\mainmatter
\sloppy

%_______________________________________________________________________________
% \include{...}
\chapter{Introduction}

{\em Dieses Dokument richtet sich an Studierende am Fachbereich 08 im 
Studiengang Bachelor of Science (Physik). Sie finden hier Beispiele 
für eine mögliche Gliederung Ihrer Arbeit und Hinweise zur 
Strukturierung des Inhalts. Selbstverständlich sollen Sie diese 
Gliederung nach den Gegebenheiten Ihrer Bachelorarbeit anpassen. 
Besprechen Sie rechtzeitig mit Ihrem Betreuer, ob Ihr Entwurf sinnvoll 
ist. Holen Sie sich auch Anregungen zur Gestaltung von Abschlussarbeiten 
aus der Literatur (siehe z.\ B.\ \verb|\cite{EbelBliefert}|).
\medskip

Sofern Sie sich dazu entscheiden, Ihr Dokument in \LaTeX\ zu erstellen, 
können Sie diese Datei als Vorlage verwenden. Fast die gesamte 
Literatur in der Physik verwendet \LaTeX, vor allem wegen der 
ausgezeichneten Möglichkeiten für das Formelschreiben.
}
\bigskip

In der Einleitung Ihrer Bachelorarbeit sollte das Thema der Arbeit 
möglichst allgemeinverständlich eingeführt werden. Gehen Sie 
dabei auch auf das weitere Umfeld der Arbeit ein und erläutern Sie, 
warum Aufgabenstellung und Herangehensweise interessant sind. Auch 
die weitere Gliederung kann angesprochen werden, um dem Leser einen 
ersten überblick über den nachfolgenden Text zu geben.

%_______________________________________________________________________________
\chapter{Main Part}

Die typische Gliederung einer Bachelorarbeit könnte so aussehen, 
wie im folgenden dargestellt. 
\medskip

Verwenden Sie aussagekräftige Kapitelüberschriften, also zum 
Beispiel {\em Aufbau eines Teilchenbeschleunigers} statt 
{\em Versuchsaufbau}.


%_______________________________________________________________________________
\section{Theoretical Foundations}

Beschreiben Sie bei einer experimentellen Arbeit die wesentlichen 
theoretischen Grundlagen und in jedem Fall den Stand der Forschung.

\section{Setup}

Wenn Sie an einem experimentellen Thema arbeiten, beschreiben Sie 
den Versuchsaufbau, auch wenn Sie an einem bereits vorhandenen 
Versuch arbeiten, soweit dies für Ihre spezielle Fragestellung 
relevant ist. 

\section{Method}

Entsprechend kann es bei einer theoretischen Arbeit sinnvoll sein,
die Lösungsmethoden in einem eigenen Kapitel zu beschreiben.

\section{Results}

Hauptteil Ihrer Arbeit ist das Kapitel (oder die Kapitel) mit den 
Ergebnissen. Bei einer theoretischen Arbeit kann damit auch 
die Herleitung von Formeln oder die Beschreibung eines Computerprogramms 
gemeint sein.

%_______________________________________________________________________________
\chapter{Conclusion}

In der Zusammenfassung sollten Sie in knapper Form die Aufgabenstellung 
und die wichtigsten Ergebnisse rekapitulieren. Es ist für die 
Gutachter hilfreich, wenn Sie ausdrücklich beschreiben, worin 
Ihre eigenen Beiträge liegen. Scheuen Sie sich auch nicht davor 
auszusprechen, welche Untersuchungen durch die Zeitbegrenzung der 
Bachelorarbeit nicht möglich waren und nutzen Sie dies als 
überleitung zu einem Ausblick auf mögliche weitergehende 
Arbeiten an der Aufgabenstellung.

%_______________________________________________________________________________
\appendix

% \chapter{Appendix}

\chapter{Tables and Figures}

In der Regel sind die in Tabellen und Abbildungen enthalten Informationen 
so wichtig, dass sie im Hauptteil der Arbeit erscheinen sollten. Unter 
Umständen sind aber ergänzende Tabellen und Abbildungen gut in einem 
Anhang aufgehoben. Wie im Hauptteil sollten Sie auch hier darauf achten, 
dass die in Tabellen und Figuren (siehe Abb.\ \ref{Abb:1}) dargestellte 
Information im Text angesprochen wird und selbsterklärende Legenden 
vorhanden sind.
\medskip

% \begin{figure}[h]
% \begin{center}
% \includegraphics[scale=0.8]{BA-Abbildung1.pdf}
% \end{center}
% \caption{\label{Abb:1}
% Feynmandiagramm für eine typische Einschleifen-Korrektur zur 
% Produktion von sieben Jets in der $e^+e^-$-Vernichtung (entnommen 
% aus \cite{thepnews}, mit Zustimmung der Autoren).
% } 
% \end{figure}


%_______________________________________________________________________________
\chapter{Weiterführende Details zur Arbeit}

Manch wichtiger Teil Ihrer tatsächlichen Arbeit ist zu technisch 
und würde den Hauptteil des Textes unübersichtlich machen, 
beispielsweise wenn es um die Details des Versuchsaufbaus in einer 
experimentellen Arbeit oder um den für eine numerische Auswertung 
verwendeten Algorithmus geht. Dennoch ist es sinnvoll, entsprechende 
Beschreibungen in einem Anhang Ihrer Bachelorarbeit aufzunehmen. 
Insbesondere für zukünftige Arbeiten, die an Ihre Bachelorarbeit 
anschlie{\ss}en, sind dies manchmal hilfreiche Informationen.

%_______________________________________________________________________________
\chapter{References}

Machen Sie genaue Angaben, so dass die verwendeten Literaturstellen 
eindeutig identifiziert und aufgefunden werden können.
Bei Lehrbüchern \cite{Weinberg:1995mt} ist es sinnvoll, 
den Titel anzugeben, eventuell auch die Ausgabe. Bei Artikeln in 
Fachzeitschriften \cite{Moch:2001zr} ist es üblich, nur die 
gebräuchlichen Abkürzungen für den Titel der Zeitschrift, Band, 
Erscheinungsjahr und Seite anzugeben. Unter Umständen kann es auch 
sinnvoll sein, im Internet aufgefundene Informationsquellen anzugeben, 
zum Beispiel für Software \cite{LoopTools} oder zu den Details von 
Ergebnissen gro{\ss}er experimenteller Kollaborationen. Es ist 
selbstverständlich, dass Sie auch Bachelor- \cite{BA:Freund}, 
Diplom- oder Doktorarbeiten angeben, wenn Sie diese in Ihrer eigenen 
Arbeit verwendet haben.
\medskip

Im folgenden Beispiel werden die in der Datei \texttt{h-physrev3.bst} 
enthaltenen Anweisungen als Stilvorlage verwendet. Andere 
Möglichkeiten für die Gestaltung eines Literaturverzeichnisses 
findet man im Internet: \url{http://janeden.net/bibliographien-mit-latex}.

\printbibliography

% \renewcommand{\bibname}{Literaturverzeichnis} 
% \bibliographystyle{h-physrev3}
% \begin{thebibliography}{99}

% %\cite{thepnews}
% \bibitem{EbelBliefert}
% H.\ F.\ Ebel, C.\ Bliefert, 
%   ``Bachelor-, Master- und Doktorarbeit: Anleitungen für den 
%   naturwissenschaftlich-technischen Nachwuchs,''
%   Wiley-VCH, Weinheim (2009). 
% \bibitem{thepnews}
%   S.~Becker, D.~Götz, C.~Reuschle, C.~Schwan, S.~Weinzierl,  
%   \url{http://wwwthep.physik.uni-mainz.de/site/news/168/}.

% %\cite{Weinberg:1995mt}
% \bibitem{Weinberg:1995mt}
%   S.~Weinberg,
%   ``The Quantum theory of fields. Vol. 1: Foundations,''
%   Cambridge, UK: Univ. Pr. (1995) 609 p.

% %\cite{Moch:2001zr}
% \bibitem{Moch:2001zr}
%   S.~Moch, P.~Uwer, S.~Weinzierl,
%   %``Nested sums, expansion of transcendental functions and multiscale 
%   % multiloop integrals,''
%   J.\ Math.\ Phys.\  {43 } (2002)  3363-3386.
%   [hep-ph/0110083].

% %\cite{LoopTools}
% \bibitem{LoopTools}
%   T.~Hahn, 
%   ``The LoopTools Site,''
%   \url{http://www.feynarts.de/looptools/}.

% %\cite{BA:Freund}
% \bibitem{BA:Freund}
%   B.~Freund, 
%   Bachelorarbeit, Johannes Gutenberg-Universität Mainz, 2012.

% \end{thebibliography}

%_______________________________________________________________________________
\chapter{Thanking}

... an wen auch immer. Denken Sie an Ihre Freundinnen und Freunde, 
Familie, Lehrer, Berater und Kollegen.


\end{document}